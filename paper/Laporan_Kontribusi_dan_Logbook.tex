\documentclass[report, a4paper, 12pt]{article}
\usepackage[utf8]{inputenc}
\usepackage[indonesian]{babel}
\usepackage{geometry}
\geometry{left=3cm,right=2.5cm,top=2.5cm,bottom=2.5cm}
\usepackage{amsmath}
\usepackage{amssymb}
\usepackage{graphicx}
\usepackage{booktabs}
\usepackage{float}
\usepackage{hyperref}
\usepackage{fancyhdr}
\usepackage{longtable}

\pagestyle{fancy}
\fancyhf{}
\rhead{\small Logbook \& Kontribusi}
\lhead{\small Kelompok 3 - Kriptografi Terapan}
\cfoot{\thepage}

\begin{document}

\begin{titlepage}
    \centering
    \vspace*{1cm}
    {\large \textbf{LAPORAN LOGBOOK DAN KONTRIBUSI PROYEK AKHIR}}\\[0.5cm]
    {\large \textbf{KRIPTOGRAFI TERAPAN}}\\[1.5cm]
    
    \includegraphics[width=0.4\textwidth]{grafik_output/flowchart.png}\\[1.5cm] % Menggunakan flowchart sebagai ilustrasi sementara
    
    \textbf{Judul Proyek:}\\
    Analisis Komparatif Performa dan Keamanan Algoritma Hash SHA-256, SHA-3, dan BLAKE2 pada Lingkungan Python\\[1.5cm]
    
    \textbf{Disusun Oleh (Kelompok 3):}\\[0.5cm]
    \begin{tabular}{ll}
        Falito Eriano Nainggolan & (Ketua Kelompok) \\
        Hinggil Parahita & (Anggota) \\
        Raffelino Hizkia Marbun & (Anggota) \\
        Yosapat Nainggolan & (Anggota) \\
    \end{tabular}\\[2cm]
    
    \textbf{Tingkat II Rekayasa Keamanan Siber A}\\
    \textbf{Politeknik Siber dan Sandi Negara}\\
    \textbf{Februari 2026}
\end{titlepage}

\section{Pendahuluan}
Laporan ini berisi dokumentasi aktivitas harian (logbook) dan rincian kontribusi masing-masing anggota kelompok dalam pengerjaan proyek akhir mata kuliah Kriptografi Terapan. Proyek ini berfokus pada evaluasi kinerja dan difusi keamanan fungsi hash SHA-256, SHA-3, dan BLAKE2.

\section{Logbook Aktivitas}
Berikut adalah rincian aktivitas yang dilakukan selama durasi proyek:

\begin{longtable}{|p{2.5cm}|p{9cm}|p{3cm}|}
    \hline
    \textbf{Tanggal} & \textbf{Deskripsi Aktivitas} & \textbf{Status} \\ \hline
    \endfirsthead
    \hline
    \textbf{Tanggal} & \textbf{Deskripsi Aktivitas} & \textbf{Status} \\ \hline
    \endhead
    27 Jan 2026 & Inisialisasi ide proyek dan pemilihan algoritma (SHA-256, SHA-3, BLAKE2). & Selesai \\ \hline
    28 Jan 2026 & Pembuatan skrip benchmarking dasar menggunakan Python \texttt{hashlib}. & Selesai \\ \hline
    01 Feb 2026 & Eksekusi benchmark awal dan pengumpulan data mentah (\textit{Raw Data}). & Selesai \\ \hline
    03 Feb 2026 & Analisis statistik menggunakan ANOVA dan optimasi grafik visualisasi. & Selesai \\ \hline
    10 Feb 2026 & Penyusunan draf awal laporan (Bab 1 - Bab 3). & Selesai \\ \hline
    15 Feb 2026 & Penulisan Bab 4 (Hasil dan Pembahasan) serta Bab 5 (Kesimpulan). & Selesai \\ \hline
    18 Feb 2026 & Audit teknis pertama: Perbaikan sitasi dan standarisasi hardware acceleration notation. & Selesai \\ \hline
    22 Feb 2026 & Audit teknis kedua: Sinkronisasi terminologi SHA-NI dan efisiensi CPU. & Selesai \\ \hline
    23 Feb 2026 & \textit{Quality Assurance} akhir: Penyesuaian desimal koma, terminologi SHA-3, dan pembersihan kode. & Selesai \\ \hline
\end{longtable}

\newpage
\section{Matriks Kontribusi Anggota}
Berikut adalah pembagian tugas dan kontribusi spesifik setiap anggota kelompok:

\begin{table}[H]
    \centering
    \begin{tabular}{@{}lp{10cm}@{}}
        \toprule
        \textbf{Nama Anggota} & \textbf{Kontribusi Utama} \\ \midrule
        Falito Eriano N. & Manajemen proyek, integrasi naskah akhir, perancangan skenario benchmark, dan analisis statistik ANOVA. \\ \midrule
        Hinggil Parahita & Pengembangan skrip visualisasi data (\textit{matplotlib/seaborn}), pengujian SAC (Avalanche effect), dan penulisan Bab 4. \\ \midrule
        Raffelino Hizkia M. & Studi literatur (\textit{Related Works}), pengelolaan referensi (BibTeX), dan penulisan Bab 1 \& 2. \\ \midrule
        Yosapat Nainggolan & Persiapan dataset (1 MB - 1 GB), analisis konsumsi sumber daya CPU (\textit{psutil}), dan penulisan Bab 3 \& 5. \\ \bottomrule
    \end{tabular}
    \caption{Matriks Peran dan Tanggung Jawab Anggota Kelompok}
\end{table}

\section{Penutup}
Seluruh anggota kelompok telah memberikan kontribusi maksimal sesuai dengan perannya masing-masing. Proyek ini berhasil diselesaikan tepat waktu dengan standar akademik yang diharapkan.

\vspace{2cm}
\begin{flushright}
    Bogor, 23 Februari 2026\\[1.5cm]
    \textbf{Falito Eriano Nainggolan}\\
    (Ketua Kelompok)
\end{flushright}

\end{document}
