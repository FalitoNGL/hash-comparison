

\section{Hasil dan Pembahasan}
\label{sec:hasil}

Bagian ini memaparkan hasil evaluasi komparatif dari ketiga algoritma hash berdasarkan metrik \textit{throughput}, efisiensi sumber daya, dan kualitas keamanan kriptografis.

\subsection{Rekapitulasi Data Eksperimen}
Tabel \ref{tab:results} menyajikan metrik rata-rata pada korpus maksimal (1 GB) untuk mengilustrasikan performa beban tinggi. Walaupun observasi untuk 1 MB, 10 MB, dan 100 MB tidak ditabulasi terpisah demi simplisitas, pola \textit{throughput} (SHA-256 > BLAKE2 > SHA-3) tetap konsisten pada seluruh ukuran data uji.

\begin{table}[H]
	\centering
	\caption{Hasil Eksekusi Referensi Lingkungan Asali (Default)}
	\label{tab:results}
	\resizebox{\columnwidth}{!}{%
	\begin{tabular}{|l|c|r|c|c|}
		\hline
		\textbf{Algoritma} & \textbf{Ukuran} & \textbf{Throughput} & \textbf{Efisiensi CPU (s/GB)} & \textbf{Difusi} \\ \hline
		SHA-256 & 1 GB & 464,00 MB/s & 1,8 & 50,8\% \\ \hline
		BLAKE2  & 1 GB & 172,00 MB/s & 4,1 & 48,2\% \\ \hline
		SHA-3   & 1 GB & 124,00 MB/s & 5,7 & 52,7\% \\ \hline
	\end{tabular}%
	}
\end{table}

\begin{table}[H]
	\centering
	\caption{Statistik Deskriptif Kinerja \textit{Throughput} ($\mu \pm \sigma$)}
	\label{tab:desc_stats}
	\resizebox{\columnwidth}{!}{%
	\begin{tabular}{|l|l|r|r|r|}
		\hline
		\textbf{Algoritma} & \textbf{Ukuran File} & \textbf{Mean (MB/s)} & \textbf{Std. Dev ($\sigma$)} & \textbf{95\% CI} \\ \hline
		SHA-256 & 10 MB & 568,0 & 14,2 & [562,7; 573,2] \\ \hline
		SHA-256 & 1 GB  & 464,0 & 12,8 & [459,1; 468,8] \\ \hline
		BLAKE2  & 10 MB & 247,0 & 8,5  & [243,8; 250,1] \\ \hline
		BLAKE2  & 1 GB  & 172,0 & 7,2  & [169,3; 174,6] \\ \hline
		SHA-3   & 10 MB & 175,0 & 6,4  & [172,6; 177,3] \\ \hline
	\end{tabular}%
	}
\end{table}

\subsection{Analisis Kinerja Kecepatan (Throughput) dan Pengaruh Mikroarsitektur}
Pengujian pada lingkungan terkontrol menunjukkan keunggulan \textit{throughput} oleh algoritma SHA-256. Seperti diilustrasikan pada Gambar \ref{fig:throughput} dan dijabarkan pada Tabel \ref{tab:desc_stats}, eksekusi CPython pada dataset 10 MB memampukan SHA-256 mencapai rata-rata 568 MB/s. Kecepatan ini 2,3 kali lebih tinggi dari BLAKE2 (247 MB/s) dan 3,2 kali melampaui SHA-3 (175 MB/s). Observasi ini secara spesifik merefleksikan performa pada lingkungan uji yang dipengaruhi oleh lapisan akselerasi instruksi perangkat keras.

\begin{figure}[H]
	\centering
	\includegraphics[width=\linewidth]{grafik_output/01_throughput_comparison.png}
	\caption{Perbandingan \textit{Throughput} (MB/s) Eksekusi Asali}
	\label{fig:throughput}
\end{figure}

Modul \texttt{hashlib} pada CPython mendelegasikan instruksi kriptografis pada ekstensi \textbf{Intel SHA Extensions (SHA-NI)} melalui rutin binari OpenSSL \textit{backend}. Untuk mengevaluasi signifikansi dukungan ini, pengujian diulang dengan menonaktifkan ekstensi melalui parameter \textit{environment} (\verb|OPENSSL_ia32cap="~0x20000000"|), kendati pemastian \textit{flag} instruksi \textit{assembly} murni pada level registri kompilator tidak dilakukan. Pada skenario isolasi perangkat keras tersebut, laju \textit{throughput} SHA-256 menurun hingga berada di bawah rata-rata konstan BLAKE2. Hasil observasional ini menunjukkan bahwa kinerja tingkat sistem pada lapisan \textit{interpreter} CPython sangat bergantung pada ketersediaan instruksi perangkat keras lokal, sehingga tidak serta-merta mencerminkan kompleksitas asimptotik algoritmanya secara universal.

\subsection{Validasi Signifikansi Statistik (Two-Way Factorial ANOVA)}
Analisis statistik digunakan untuk mengevaluasi parameter \textit{Interaction Effect} persilangan antara algoritma hash dan ukuran dataset guna mengevaluasi apakah perbedaan metrik \textit{throughput} melampaui variabilitas normal pengukuran.

Terdapat efek interaksi Algoritma $\times$ Ukuran File yang signifikan terhadap \textit{throughput}, $F(6, 348) = 210,45, p < 0,001, \eta_p^2 = 0,88$. Nilai ini mengindikasikan bahwa dalam konteks model pengujian ini, efek gabungan algoritma dan ukuran data mampu menjelaskan 88\% varians pada \textit{throughput}. Berdasarkan uji perbandingan ganda Bonferroni (\textit{post-hoc pairwise confidence 95\%}), selisih \textit{throughput} rata-rata antara SHA-256 dan BLAKE2 pada ukuran 1 GB memiliki limit positif sebesar 292 MB/s ($p < 0,001$). Interpretasi signifikansi secara statistik ini terbatas pada ruang lingkup arsitektur sistem CPython x86\_64, dan tidak merepresentasikan kinerja pada arsitektur lain (seperti ARM64) maupun jenis \textit{runtime} berbeda (seperti PyPy).

\subsection{Analisis Efisiensi: Waktu Utilisasi CPU vs Kecepatan}
Gambar \ref{fig:efficiency} memetakan pola distribusi \textit{throughput} terhadap rasio durasi waktu CPU (\textit{User Time} + \textit{System Time}). SHA-256 mencatat waktu terendah sebesar 1,8 \textit{CPU-second} per GB. Sebaliknya, komputasi SHA-3 menghabiskan rata-rata 5,7 \textit{CPU-second} per GB. Perbedaan ini menunjukkan peningkatan beban komputasi pada eksekusi non-terakselerasi dibandingkan eksekusi instruksi natif (\textit{hardware offloading}).\vspace{-0.1cm}

\begin{figure}[H]
	\centering
	\includegraphics[width=\linewidth]{grafik_output/02_efficiency_matrix.png}
	\caption{Matriks Efisiensi: \textit{Throughput} vs Penggunaan CPU}
	\label{fig:efficiency}
\end{figure}

\subsection{Evaluasi Properti Difusi (Strict Avalanche Criterion)}
Pengujian proksi SAC berskala besar ($N=10.000$ entitas biner) memvalidasi struktur \textit{Sponge} dan \textit{Merkle–Damgård} masing-masing algoritma. Model rasio inversi ideal mewajibkan kesesuaian dispersi dengan Distribusi Binomial $B(256, 0,5)$, ditandai oleh rata-rata ideal jarak difusi 128 bit (\textit{variance} $\approx 64$) \cite{webster1986}.

\begin{figure}[H]
	\centering
	\includegraphics[width=\linewidth]{grafik_output/03_avalanche_consistency.png}
	\caption{Deviasi Distributif \textit{Avalanche} \textit{Hamming Distance} (N=10.000)}
	\label{fig:avalanche}
\end{figure}

Secara taksonomi empiris, ketiga algoritma berkinerja dalam rentang kriteria Binomial. SHA-3 menampilkan distribusi aktual rataan diskrit 128,02 (Varians = 64,15), sementara SHA-256 dan BLAKE2 masing-masing merekam rataan 128,00 dan 127,98. Hasil uji \textit{Chi-Square Goodness-of-Fit} menunjukkan tidak terdapat perbedaan signifikan antara sebaran SHA-3 melawan model teoretikal ($\chi^2(25) = 28,4, p = 0,28$; kondisi serupa ditemukan pada algoritma lainnya, $p > 0,10$). Evaluasi makroskopik ini menunjukkan bahwa properti difusi (SAC) berfungsi konsisten tanpa anomali struktural melintasi lapisan \textit{interpreter} CPython.