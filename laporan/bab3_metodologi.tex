

\section{Metodologi Penelitian}
\label{sec:metodologi}

Penelitian ini menggunakan pendekatan eksperimental kuantitatif. Rangkaian pengujian dirancang untuk mengukur kinerja kriptografis dengan mengontrol variabel sistem, sehingga membatasi interferensi eksternal dan memungkinkan reproduksibilitas pengujian (\textit{reproducible}).

\subsection{Konfigurasi Lingkungan Pengujian}
Untuk meminimalisir bias dari \textit{overhead} sistem operasi, spesifikasi perangkat keras dan tumpukan perangkat lunak (\textit{software stack}) dibakukan pada konfigurasi berikut:
\begin{itemize}
	\item \textbf{Prosesor (CPU):} 12th Gen Intel(R) Core(TM) i5-12450H (Arsitektur Alder Lake, 8 \textit{Physical Cores}, 12 \textit{Threads}).
	\item \textbf{Memori (RAM):} 16.0 GB DDR4.
	\item \textbf{Sistem Operasi:} Windows 11 64-bit (Build 10.0.26200).
	\item \textbf{Lingkungan Eksekusi:} Python versi 3.14.2 menggunakan \texttt{hashlib} sebagai \textit{wrapper} OpenSSL. Untuk menguji kausalitas perangkat keras, eksekusi dipisah menjadi dua kontrol (\textit{Control Groups}):
    \begin{itemize}
        \item \textbf{Hardware-Accelerated (Default):} Mengizinkan pemanggilan utilitas \textit{Intel Secure Hash Algorithm Extensions} (SHA-NI).
        \item \textbf{Software-Only (Isolated):} Memaksa modul OpenSSL memintas instruksi kriptografi spesifik melalui pemasangan variabel OS \verb|OPENSSL_ia32cap="~0x20000000"| saat \textit{runtime}.
    \end{itemize}
\end{itemize}

\subsection{Persiapan Dataset Uji}
Dataset pengujian menggunakan \textit{pseudo-random byte sequence} yang diekstraksi dari fungsi \texttt{os.urandom}. Data di-generasi ke dalam ruang memori RAM (\textit{in-memory payload}) sebelum \textit{benchmarking} dijalankan, demi menghindari hambatan \textit{disk I/O}. Varian ukuran data yang diuji adalah 1 MB, 10 MB, 100 MB, dan 1 GB.

\subsection{Skenario dan Tahapan Pengujian}
Protokol pengujian mengadopsi standar \textit{benchmarking} dari Raj Jain \cite{jain1991} yang diilustrasikan pada diagram alir di Gambar \ref{fig:flowchart}.

\begin{figure}[H]
	\centering
	\includegraphics[width=0.7\linewidth]{grafik_output/flowchart.png}
	\caption{Diagram Alir (Flowchart) Skenario Pengujian Algoritma Hash}
	\label{fig:flowchart}
\end{figure}

Tahapan pengujian dilakukan secara sekuensial untuk setiap algoritma:
\begin{enumerate}
	\item \textbf{\textit{Warm-up Phase}:} Setiap algoritma mengeksekusi hash sebanyak 2 kali tanpa dilakukan pencatatan hasil. Langkah ini bertujuan menginisialisasi tabel rotasi di dalam \textit{cache L1/L2 silikon CPU} sebelum pencatatan \texttt{perf\_counter\_ns} dimulai. Walaupun eksperimen tidak memverifikasi secara matematis pencapaian \textit{steady-state} dalam 2 iterasi, langkah ini digunakan untuk mengurangi potensi \textit{cold-start bias} pada awal iterasi.
	\item \textbf{\textit{Evaluation Phase}:} Setiap kombinasi algoritma dan ukuran file diuji sebanyak 30 iterasi independen (dengan payload acak baru pada tiap putaran). Harus diakui bahwa eksperimen repetitif pada mesin yang sama memiliki potensi autokorelasi temporal (seperti \textit{thermal throttling}), namun pengacakan \textit{payload} digunakan untuk mengurangi potensi variabilitas tersebut.
\end{enumerate}

\subsection{Logika Implementasi Benchmark}
Kalkulasi \textit{throughput} menggunakan \texttt{time.perf\_counter\_ns()}. Fungsi ini mengakses \textit{hardware monotonic clock} pada prosesor untuk meminimalkan deviasi pencatatan standar berbasis \textit{wall-clock}. Konsumsi sumber daya diukur terisolasi melalui \texttt{psutil.Process().cpu\_times()}, bukan melaui persentase utilisasi sistem global. Parameter ini menjumlahkan \textit{User Time} dan \textit{System Time} untuk merekam beban CPU aplikasi secara independen guna memitigasi distorsi dari proses berlatar OS (\textit{system interrupts}). Kendati *jitter* dari *scheduler* OS pada tingkat milidetik berpotensi terjadi, skala korpus data pengujian yang besar menekan signifikansi *noise* tersebut.

\subsection{Evaluasi Properti Difusi (Strict Avalanche Criterion)}
Pengujian dilakukan untuk mengukur kualitas penyebaran bit pada keluaran hash berbasis \textit{Strict Avalanche Criterion} \cite{webster1986}. Metrik probabilitas di pengujian Avalanche ini difokuskan secara independen pada kualitas difusi. Evaluasi SAC memproyeksikan stabilitas pengacakan data semata, dan bukan representasi ukur matematis langsung bagi ketahanan kolisi (\textit{collision resistance}) ataupun ketahanan prabayangan (\textit{preimage resistance}).

Dalam pengujian skala besar, $N=10.000$ blok masukan acak mandiri berkapasitas 64-byte di-induksasikan skema pembalikan (\textit{bit flipping}) tepat pada satu unit acak bit tunggal secara beraturan silang (\textit{uniform independent bit inverison}). Perbedaan selisih di antara \textit{hash digest} awal dan inversi diekstraksi ke observasi jarak diskrit berupa \textit{Hamming Distance}. Evaluasi ini berbasis rata-rata global, dan tidak ditujukan untuk menguji korelasi antar bit tunggal (\textit{strict bit-independence}) pada pengujian kriptoanalisis mendalam. Validasi persentual sebaran bit diekstrak menggunakan Uji \textit{Chi-Square Goodness-of-Fit}, dikomparasikan pada referensi kurva standar \textit{Binomial Distribution} $B(256, 0.5)$ (kelas dengan frekuensi ekspektasi $<5$ diagregasi untuk memenuhi asumsi uji Pearson dengan $df=25$).

\subsection{Rancangan Analisis Statistik}
Analisis komparatif menggunakan model \textit{Two-Way Factorial Analysis of Variance} (ANOVA). Rancangan eksperimen faktorial ini menggunakan dua parameter utama: Jenis Algoritma (3 level: SHA-256, SHA-3, BLAKE2) dan Ukuran File Dataset (4 level: 1MB, 10MB, 100MB, 1GB). Pengamatan didapatkan melalui 30 iterasi pengujian per kombinasi perlakuan, menghasilkan total observasi $N=360$ di luar proses pengabaian \textit{warm-up}. 

Model ANOVA berfokus mengevaluasi signifikansi efek (\textit{Interaction Term: Algoritma $\times$ Ukuran File}) terhadap metrik \textit{throughput}. Normalitas residual diasumsikan terpenuhi melalui Teorema Limit Pusat (\textit{Central Limit Theorem}) berkat ukuran sampel ($n=30$). Homogenitas varians diasumsikan mematuhi prasyarat parametrik karena besaran observasi sel sampel sama rata (\textit{balanced design}), meskipun uji eksplisit (\textit{Levene's Test}) tidak disertakan dalam laporan ini. Ukuran pengaruh dilaporkan secara kuantitatif menggunakan \textit{partial eta-squared} ($\eta_p^2$). Perbedaan rata-rata kelompok dievaluasi melalui uji pembandingan berpasangan interval kepercayaan Bonferroni 95\%.