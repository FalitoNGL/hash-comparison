% =================================================================
% BAB 5: KESIMPULAN DAN SARAN (VERSI FIX & TEKNIS TINGKAT LANJUT)
% =================================================================

\section{Kesimpulan dan Saran}
\label{sec:kesimpulan}

\subsection{Kesimpulan}
	Studi ini mengevaluasi kinerja \textit{throughput}, efisiensi CPU, dan properti difusi algoritma hash SHA-256, SHA-3, dan BLAKE2 pada ekosistem terinterpretasi CPython (x86\_64, Windows 11). Pengujian menunjukkan \textit{throughput} rata-rata SHA-256 (568 MB/s) terukur 2,3 kali lebih tinggi dibandingkan BLAKE2. Disparitas kinerja ini berkorelasi langsung dengan ketersediaan instruksi \textit{Intel SHA Extensions} (SHA-NI). Menonaktifkan ekstensi perangkat keras ini secara konsisten menurunkan kecepatan eksekusi SHA-256 hingga di bawah kinerja BLAKE2. Terkait efisiensi alokasi siklus CPU, SHA-256 mencatat utilisasi durasi terendah (1,8 \textit{CPU-second} per GB), sedangkan arsitektur \textit{Sponge} SHA-3 merekam beban rata-rata 5,7 \textit{CPU-second} per GB. Pengujian properti difusi (\textit{Strict Avalanche Criterion}, $N=10.000$) mengindikasikan bahwa abstraksi instruksi tingkat lapisan bawah tidak mendistorsi kualitas penyebaran bit, dengan deviasi yang tetap sesuai target Distribusi Binomial ($p > 0,05$). Berdasarkan batasan uji ini, komparasi kinerja kriptografis pada lingkungan interpreter CPython sangat tertaut pada integrasi instruksi perangkat keras (\textit{hardware integration}) dan tidak serta-merta merepresentasikan efisiensi asimptotik algoritma tersebut.

\subsection{Ancaman terhadap Validitas (Threats to Validity) dan Keterbatasan}
\begin{enumerate}
	\item \textbf{Validitas Eksternal:} Kesimpulan kinerja ini secara eksklusif dibatasi pada implementasi modul C dalam \textit{interpreter} CPython generasi 3.x pada arsitektur perangkat keras x86\_64 (Intel). Temuan analisis statistik ini tidak berlaku mutlak dan tidak merepresentasikan kinerja pada \textit{runtime} JIT seperti PyPy maupun arsitektur kumpulan instruksi (\textit{Instruction Set Architecture}) yang berbeda seperti ARM64 (misal Apple M-Series).
	\item \textbf{Limitasi Skala Data:} Pengujian dikontrol ketat pada korpus deterministik rentang makro (1 MB hingga 1 GB). Pola komparasi performa pada pemrosesan \textit{micro-payloads} ekstrem (contoh: beban REST API $<1$ KB) akan menghasilkan rasio inefisiensi arsitektural berbeda yang didominasi \textit{overhead interpreter intialization} dibandingkan waktu pemrosesan kriptografis murni.
\end{enumerate}

\subsection{Saran untuk Penelitian Selanjutnya (Future Works)}
Mengacu pada keterbatasan dan temuan anomali dalam penelitian ini, penulis merekomendasikan peta jalan penelitian selanjutnya:
\begin{enumerate}
	\item \textbf{Isolasi Interpreter/Kompiler:} Melakukan evaluasi ulang (\textit{re-evaluation}) menggunakan bahasa pemrograman tingkat rendah seperti C11 atau Rust yang dikompilasi secara manual menggunakan \textit{flag} optimasi khusus (misalnya \texttt{-mavx2}) untuk mengukur performa mentah (\textit{raw performance}) BLAKE2 tanpa adanya \textit{overhead} dari \textit{interpreter} Python.
	\item \textbf{Arsitektur Perangkat Keras Berbeda:} Melakukan pengujian pada arsitektur ARM64 (seperti Apple Silicon M-Series atau Raspberry Pi) di mana set instruksi mikrokode kriptografinya berbeda secara fundamental dengan arsitektur x86\_64 milik Intel.
	\item \textbf{Analisis Energi Silikon:} Menggunakan instrumen \textit{hardware} untuk mengukur konsumsi daya listrik dalam satuan Watt atau Joule per-hash, guna menentukan algoritma mana yang paling efisien untuk ekosistem \textit{Internet of Things} (IoT) bertenaga baterai.
\end{enumerate}